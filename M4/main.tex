\documentclass{beamer}

\usepackage[utf8]{inputenc}
\usepackage[polish]{babel}
\usepackage[lf]{berenis}
\usepackage[LY1]{fontenc}


\usepackage{utopia}

\usepackage{listings}
\usepackage[dvipsnames]{xcolor}

 \lstset{
basicstyle=\scriptsize\sffamily\color{black},
frame=single,
numbers=left,
numbersep=5pt,
numberstyle=\tiny\color{gray},
showspaces=false,
showstringspaces=false,
tabsize=1
}
 
\definecolor{codegreen}{rgb}{0,0.6,0}
\definecolor{codegray}{rgb}{0.5,0.5,0.5}
\definecolor{codepurple}{rgb}{0.58,0,0.82}
\definecolor{backcolour}{rgb}{0.95,0.95,0.92}

\lstdefinestyle{mystyle}{
    backgroundcolor=\color{backcolour},   
    commentstyle=\color{codegreen},
    keywordstyle=\color{magenta},
    numberstyle=\tiny\color{codegray},
    stringstyle=\color{codepurple},
    basicstyle=\ttfamily\footnotesize,
    breakatwhitespace=false,         
    breaklines=true,                 
    captionpos=b,                    
    keepspaces=true,                 
    numbers=left,                    
    numbersep=5pt,                  
    showspaces=false,                
    showstringspaces=false,
    showtabs=false,                  
    tabsize=2
}

\lstdefinelanguage{Kotlin}{
  comment=[l]{//},
  commentstyle={\color{gray}\ttfamily},
  emph={delegate, filter, first, firstOrNull, forEach, lazy, map, mapNotNull, println, return@},
  emphstyle={\color{red}},
  identifierstyle=\color{black},
  keywords={abstract, actual, as, as?, break, by, class, companion, continue, data, do, dynamic, else, enum, expect, false, final, for, fun, get, if, import, in, interface, internal, is, null, object, override, package, private, public, return, set, super, suspend, this, throw, true, try, typealias, val, var, vararg, when, where, while},
  keywordstyle={\color{blue}\bfseries},
  morecomment=[s]{/*}{*/},
  morestring=[b]",
  morestring=[s]{"""*}{*"""},
  ndkeywords={@Deprecated, @JvmField, @JvmName, @JvmOverloads, @JvmStatic, @JvmSynthetic, Array, Byte, Double, Float, Int, Integer, Iterable, Long, Runnable, Short, String},
  ndkeywordstyle={\color{orange}\bfseries},
  sensitive=true,
  stringstyle={\color{green}\ttfamily},
}
 
\lstset{style=mystyle}

\usetheme{CambridgeUS}
\usecolortheme{default}

\title[Kotlin]{Kotlin - Przyszłość aplikacji mobilnych}
\author{Mikołaj Macura}
\institute{RMS}
\date{2019}

%=============================================

\begin{document}

\frame{\titlepage}

%=============================================

\begin{frame}
\frametitle{Podstawowe informacje}

\begin{block}{Cechy}
Wieloparadygmatowy, Typowanie statyczne, Działa na JVM
\end{block}

\begin{block}{Rok}
2011
\end{block}

\begin{block}{Twórcy}
JetBrains
\end{block}

\begin{block}{Założenia}

\begin{itemize}
\item Prędkość kompilacji na poziomie Javy
\item Interoperacyjność z kodem Javy
\item Eliminacja błędów odwołania [null-pointer safety]
\end{itemize}
\end{block}
\end{frame}

%=============================================

\begin{frame}[fragile]

Kotlin:
\begin{lstlisting}[label={lst:example1}, language=Kotlin]
fun main(args: Array<String>) {
    val Liczba1: Int = 10
    var Liczba2: Int = 20
    val suma = Liczba1 + Liczba2
    println("Suma tych liczb to: $suma")
}
\end{lstlisting}

Java:
\begin{lstlisting}[label={lst:example2}, language=Java]
public class Dodawanie {
    public static void main(String[] args) {
        
        static int Liczba1 = 10;
        int Liczba2 = 20;
        static int suma = Liczba1 + Liczba2;
        System.out.println("Suma tych liczb to: " + suma);
    }
}
\end{lstlisting}
\end{frame}

%=============================================

\begin{frame}
\begin{figure}
    \centering
    \href{https://kotlinlang.org/}{\includegraphics[scale = 0.06]{pic/Kotlin_logo.png}}
\end{figure}

\begin{figure}
    \centering
    \begin{minipage}{0.5\textwidth}
        \centering
        \includegraphics[width=0.4\textwidth]{pic/Android_Logo.png} % first figure itself

    \end{minipage}\hfill
    \begin{minipage}{0.5\textwidth}
        \centering
        \includegraphics[width=0.4\textwidth]{pic/IDEA_Logo.png} % second figure itself
    \end{minipage}
\end{figure}

\end{frame}

%=============================================

\begin{frame}

\begin{figure}
    \centering
    \begin{minipage}{0.5\textwidth}
        \centering
        \includegraphics[width=0.4\textwidth]{pic/pinterest.png} % first figure itself
    \end{minipage}\hfill
    \begin{minipage}{0.5\textwidth}
        \centering
        \includegraphics[width=0.4\textwidth]{pic/postmates.png} % second figure itself
    \end{minipage}
\end{figure}

\begin{figure}
    \centering
    \begin{minipage}{0.5\textwidth}
        \centering
        \includegraphics[width=0.4\textwidth]{pic/uber.jpg} % first figure itself
    \end{minipage}\hfill
    \begin{minipage}{0.5\textwidth}
        \centering
        \includegraphics[width=0.4\textwidth]{pic/evernote.png} % second figure itself
    \end{minipage}
\end{figure}

\end{frame}

%=============================================

\begin{frame}{Powody migracji z Javy na Kotlina}

\begin{block}{Zalety Kotlina}
\begin{itemize}
    \item Zwięzłość
    \item Większy komfort użytkowania aplikacji
    \item System Fail-Fast
    \item Doskonała dokumentacja
    \item Wsparcie społeczności
\end{itemize}
\end{block}

\begin{figure}
    \centering
    \includegraphics[scale=0.2]{pic/andr_kotlin.png}
    \label{fig:my_label}
\end{figure}

\end{frame}

\end{document}